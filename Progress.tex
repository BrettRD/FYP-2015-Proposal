\documentclass[11pt]{article}
\usepackage{amsmath}
\usepackage{graphicx}
\usepackage{subcaption}
\usepackage{listings}
\usepackage{color}
\title{Hexacopter\\
	Progress Report}
\author{Brett Downing}
\date{}
\begin{document}
  \maketitle

  Progress Report
  
  Accurate judgement of progress to date and plan of further progress.

  \section{Preamble}
  %Buzzword Bingo
  Multicopters are increasingly popular... Simplicity... Market Saturation... rapidly changing legal situation... 
  Recent Developments in power and control electronics. 

  Much of the technology related to multicopters is applicable to most other forms of UAV.
  Agriculture, targeted crop dusting
  Mapping
  Cinematography, extreme-sport photography, 


  \section{Where we started}
  The Hexacopter is a DJI F550 frame with fancy-pants motors and ESCs
  The 2013 team  decided to use a NAZA-V1-lite flight computer. This works well for free-flight, but does not make provisions for way-point navigation or telemetry.
  The on-board server is currently doing all of the autonomous navigation processing which is appropriate for computer-vision directed flight-modes, but adds additional hurdles to way-point navigation; a problem that is already very well-solved in industrial, hobbyist, and consumer grade drones.



  \section{What we've achieved}
  We've implemented and tested the code that we were able to salvage from the previous year-groups,
  Mathematics that estimates the location of a point of interest based on certain assumptions

  We've made a proposal and ordered parts to move to the Ardupilot flight control software running on the Pixhawk flight computer.
  This move allows us to utilise the vast array of supporting software that the Ardpilot community has written including ground-stations, telemetry loggers, Smart-phone apps, Kalman navigation filters and automatic flight control tuning.

  Oddly, our tests of the previous year-groups' code have produced better results than are published in the respective papers, however the claims of the previous groups still appear grossly overstated.
  Various improvements to the hardware including landing gear, Wiring harness, enclosures

  Reverse-engineering of circuits used by the previous groups. Having talked to the previous teams, we appear to have generated better documentation about the hardware than the teams had worked with initially.


  \section{What we intend to do}


	
\end{document}

